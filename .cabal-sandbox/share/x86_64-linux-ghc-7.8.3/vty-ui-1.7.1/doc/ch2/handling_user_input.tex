\section{Handling User Input}

Many widgets in \vtyui\ can accept user input.  A widget can accept
user input if (1) it has one or more \textit{key event handlers}
attached to it and (2) if it currently has the \textit{focus}.  The
concept of focus in \vtyui\ works the same as in other user interface
toolkits: essentially, only one widget has the focus and any user
input is passed to that widget for handling.

Key event handlers can be added to any \fw{Widget a} as follows:

\begin{haskellcode}
 w <- someWidget
 w `onKeyPressed` \this key modifiers -> do
   ...
   return False
\end{haskellcode}

The handler must return \fw{IO Bool}; \fw{True} indicates that the
handler processed the key event and took action and \fw{False}
indicates that the handler declined to handle the event.  The event
handler is passed the keystoke itself along with any modifier keys
detected by the underlying Vty input processing.

Key event handlers are invoked in the order in which they are added to
the widget.  In the following example, the first handler will decline
the \fw{'q'} key event but the second one will process it:

\begin{haskellcode}
 w `onKeyPressed` \_ key _ ->
   if key == KASCII 'f' then
     (launchTheMissiles >> return True) else
     return False

 w `onKeyPressed` \_ key _ ->
   if key == KASCII 'q' then
     exitSuccess else return False
\end{haskellcode}

This functionality allows any widget to have its own "default" input
event handling while still allowing you to add custom input event
handling.

Although any widget -- even a basic text widget -- can accept input
events in this way, the events will only reach the widget if it has the
focus.  The way we manage focus is with "focus groups."
