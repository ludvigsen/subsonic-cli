\section{Groups}
\label{sec:groups}

The \fw{Group} module provides a widget for containing a group of
widgets of the same type, together with a pointer to the ``current''
widget for the group.  This can be used to embed a collection of
widgets in the interface while being able to change which of the
widgets is being displayed.  This prevents users from having to
construct new interfaces around each new widget, and the group can be
modified at runtime.

To create a group, use the \fw{newGroup} function:

\begin{haskellcode}
 g <- newGroup
\end{haskellcode}

A group contains one or more widgets of any type, although they must
all have the same type within the group.  To add widgets to a group,
use \fw{addToGroup}:

\begin{haskellcode}
 switchToT1 <- addToGroup g =<< plainText "first"
 switchToT2 <- addToGroup g =<< plainText "second"
\end{haskellcode}

The \fw{addToGroup} function returns an \fw{IO} action.  This action,
when evaluated, will change the group's currently-active widget to the
one passed to \fw{addToGroup}.  In the above example, evaluating
\fw{switchToT2} would cause group \fw{g} to show the text widget
containing \fw{"second"}.

\subsubsection{Input and Focus Events}

Group widgets relay all key events received to the currently-active
widget in the group, if any.  Focus events on the group propagate to
the currently-active widget.

\subsubsection{Growth Policy}

Group widgets act as wrappers for the widgets they contain, so they
delegate all growth policy settings from the widgets being wrapped.
